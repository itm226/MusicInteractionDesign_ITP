\documentclass{article}
\usepackage[margin = 0.75in]{geometry}
\usepackage{fancyhdr}
\usepackage{url}

\rhead{Music Interaction Design}

\title{Music Interaction Design Notes}
\author{Me, Ian}
\date{Spring Semester, 2019}


\begin{document}
	\maketitle
	
	\section*{January 30th, 2019}
	
	\subsection*{Overview}
	\textsc{From the slide-show}: ``In this class students develop interactive music projects: pieces of music that are not linear, but rather offer a multiple dimensions for listeners to explores\textemdash on their phones in a crowded subways, at an abandoned factory in Palmero, back on their couches after a long day, or at a classical concert hall."
	
	\subsection*{Website}
	\url{https://luisaph.github.io/music-interaction-design-spring-2019}
	
	\subsection*{Topics}
	
	\subsection*{Assignments}
	
	\paragraph{Assignment 1}
	Document in class work
	\paragraph{Assignment 2}
	Create your own project prompt. It cna be real, made up, or a combination. 
	``Design a participative performance with X band and for Y venue using people's iPhones/cardboard boxes" 
	``Create an interactive exhibit for the museum of Samba in Rio de Janeiro , for people to experience"
	
	\section*{February 6th, 2019}
	
	\subsection*{Elements}
	
	\subsubsection*{Rhythm}
	
	\paragraph{Beat}
	
	
\end{document}
